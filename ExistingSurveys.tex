\section{Existing Surveys}
There are many surveys covering the different aspects of Recommendation Systems. Some of them are general in nature and some are specific to a particular subject; like \cite{D7} summarizes all the Recommendation Systems that use user reviews.\cite{D5} summarizes content based RS.
\\In \cite{D5} recent trends in content based RS are reviewed.After presenting a brief history of content based recommendation systems; latest trends were described in terms of data and algorithms. In terms of data survey argued that usage of Link Open Data(LOD) has is being employed now a days to get the extra meta-data features of an item. Moreover,more content is gathered from forum, user reviews and tags; they categorized it as User Generated Data(UGD), Visual and multimedia features and Heterogeneous information networks are also discussed as a source of taking content.In terms of algorithms,
following approaches were highlighted;Meta-path based; in which a path is described corresponding to a relation between two entities,New metadata encoding; word and doc embeddings to identify latent features of text and Deep learning techniques.
\\
Ontology is a formal description of any particular domain that can be presented in form of entities, concepts objects,properties,rules and relationships.In \cite{D2} specifically those RS were focused that employed ontologies for the development of e-learning RS .At start of the survey it briefly discussed about all the possible types of RS.After providing the detailed description of ontologies and e-learner system, writer summarizes all the papers in hand in terms of ontologies used,ontology representation language and Recommended learning resources 
\\
In \cite{D7} survey a detailed study was carried on about the Recommendation System fully based upon user reviews(UR) or their efficiency has been improved using UR. In the first part of the survey the general introduction of RS was given including its basic techniques (Content-based , Rating-based collaborative filtering, Preference-based product ranking), second part of the survey discussed about the elements of the reviews used in RS that are (Frequent terms,Review topics,Feature opinions, Contextual opinions, Comparative opinions,Review emotions,Review helpfulness ); in next two sections of survey all those studies were discussed in detailed who used UR as user profiling and product profiling  for recommendations and in the last they discussed about the practical implication of their findings for five dimensions: data condition, new users, algorithm improvement, profile building, and product domain.
\\Recently the trend is shifted towards deep learning even in the field of RS and almost every one using it to build the RS for its proven and promising results.\cite{DeepLearning-Srvey} compiled all these studies. Although the paper whole sole focus is deep learning techniques and it doesn't directly mention usage of text based RS but it does includes all the work done on textual data using DL approach.\cite{p3} also summarizes the research studies of deep-learning on RS. Although in this survey the techniques are described briefly but the issues and challenges address using DL are compiled. Moreover,the domains in which these models are adopted are also presented in a structured form.  \\
There are some domain specific studies as well, like\cite{p6} is covers all the studies about research paper recommendations in the time of 1998 and 2013 and claims that there are more than 200 articles published about this particular topic.But it doesn't give any insight about the details of algorithms used to extract textual features.Similarly, \cite{p4} is focused about News Recommendations only. Although the text based approaches covering in this is very concise and short, yet it does give an overview of trend people working on News RS. The study gives an outline of Algorithmic approaches, challenges and evaluation metrics of News RS. The shortened summary of popular and publicly available data sets are also included in this work.Another Study \cite{p5} enlists the challenges and methods of News Domain but it hardly mentions any particular design on pure textual data, rather its more general in nature.

\subsection{Summary of Surveys}
As described above there are a variety of surveys and state of the art studies that compiles and present the previous work in different perspective. Here our main focus of studying all those out of these which are based on textual data and talks about textual data recommendation in one way or other, in partial or complete way. We ignored those studies which were more general in nature.
SP: Survey Process (Methodology)
\\AS : Articles Selection, articles are searched and gathered with some strategy
\\ST:Summary Table: Summary of analysis is presented
\\AT: About Text Only

\\C: Completely / Yes
\\P: Partial
\\N: Not present at all

\begin{table}[!htbp] 
\centering
\footnotesize
\def\arraystretch{1.4}%
\centering
\begin{tabular}{|p{2.1cm}|p{3.6cm}|p{2.1cm}|p{1cm}|p{1cm}|p{1cm}|p{1cm}|}
\hline
\textbf{Study} & \textbf{Major Focus} & \textbf{Coverage}  & \textbf{SP} & \textbf{AS}& \textbf{ST}& \textbf{AT}
\\
\hline 
\multirow{\cite{D7}}&User Reviews&not specified&N&N& C&C
\\
\hline 
\multirow{\cite{D5}}&Content base&2000-2019&N&N& N&P
\\
\hline 
\multirow{\cite{D2}}&Ontology based RS for e-learning systems&2005-2014&C&C& C&P
\\
\hline
\multirow{\cite{DeepLearning-Srvey}}&Deep Learning Techniques used in RS&2005-2014&C&C& P&P
\\
\hline
\multirow{\cite{p3}}&challenges and remedies of RS using Deep Learning &2013-2017&P&C& C&P
\\
\hline
\multirow{\cite{p4}}&News Recommendation Systems &2006-2015&C&C& C&P
\\
\hline
\end{tabular}

\caption{Summary of existing surveys}
\end{table}
\\
\\
