\section{Abstract}
Many platforms over the internet are producing a variety of textual data; such as news,research articles, personal blogging and user reviews. This massive textual data generation over the Internet has enhanced the need of quick searching of required information. To help in speeding up this process of filtering, Recommendation System are being developed for all those platforms where suggestions are required to made by the system to the users to help them find the object of their interest. This study aims to present an overview of techniques that can be used to extract pragmatic features and useful method to construct effective recommendation systems on textual data.The studies also comprehends the metrics used for evaluation of such systems.For compilation of this study, literature is being collected from preeminent digital repositories.The survey covers early approaches to the latest one based upon machine learning and deep learning solutions. Survey concludes that there is no one specific data-set that is used as bench mark of any domain in terms of textual data. Hence,due to inadequacy of standard data-sets, most of the time the justification of one's work is given by implementing different base-line algorithms on one's own data-set.Moreover, offline evaluation is mostly adopted because they are easy to employ and there is a genuine lack in real-time testing platforms.Furthermore, there is a clear trend shift towards deep learning techniques in recent years as compared to the past.This study shows that there are a number of evaluation metrics out there applying on a variety of features with various algorithms used. Therefore, there is no clear standardization seen.Hence the domain specific standardization of data-sets and availability of online evaluation platform are the major challenges.     
