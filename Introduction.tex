\section{Introduction}
The advancements in the digital technology, specially after the introduction of smart phones, the online data has been exploded tremendously. Social sites such as facebook and twitter are a big source of data generation. Moreover, question answering sites such as Quora and Stack overflow are also adding data in this pool. Furthermore the number of publication has been enormously increased over last few years and lastly the trend of personal blogging,which is also an account of data production. This data is of huge importance not only because it is a source of information but also because the business is now revolutionized because of this digitization.For instance, at present the largest taxi service provider Uber having revenue of 6.5 billion dollars in 2016\footnote{https://www.reuters.com/article/us-uber-tech-results/ubers-revenue-hits-6-5-billion-in-2016} , however it does not own a single taxi. Furthermore, the company has very limited physical space compared to the taxi services of the 20th century. As an alternative to that, the application software for the company plays a pivotal role in providing the service.  The largest retailer   in the world, Alibaba\footnote{https://www.statista.com/statistics/225614/net-revenue-of-alibaba/} , which is a 36.68 billion dollar, primarily operators on the web with, limited physical presence. Similarly, Amazon\footnote{https://www.statista.com/topics/846/amazon/}  is another large retailer that has revenue of 177.87 billion dollars and 310 million active customers that shop online. It is a widely established that customers interested in utilizing online services are growing rapidly.
\newline
Typically, a single web-based business site can have several hundred products or services that are grouped into various classes. In the presence of these large numbers of products or services, a customer is required to navigate through them or search for the products or services to find the appropriate one. For instance, consider an online bookstore having several thousand books. In order to find a book of interest, the customer is required to search through the whole website to find the book of his interest. Similarly, an online electronic store may have several thousand products, and a customer may find it hard to navigate and find the most relevant product or interest. To address that problem, recommender systems have been developed \cite{RSSurvey}. A recommender system keeps track of customer’s profile and based on their interest recommends a product or a service\cite{SemanticRelationForContentBased}. Specifically, a recommender system is special kind of software tools and techniques that give suggestions to a user about specific items that can be of their interest. These suggestions can occur in any domain ranging from buying a product to listening to a song or watching a movie. The benefits of using recommender systems are manifolds. a) it reduces our effort of searching and navigating products, b) it is an effective marketing and advertising tool c) it brings customer satisfaction to business\cite{handbook}. 
\newline In a typical recommender system, the recommendation problem is twofold, i.e., (i) estimating a prediction for an individual item or (ii) ranking items by prediction (Sarwar et al. 2001). And there are various types of approaches to achieve this task. Most popular among them are Collaborative filtering(CF) and Content based (CB).CF recommender systems produce recommendations to its users based on inclinations of other users with similar tastes whereas Content-based recommender systems generate recommendations based on similarities of new items to those that the user liked in the past by exploiting the descriptive characteristics of items. However recently both these techniques are used together in hybrid fashion and showed promising results by overcoming the disadvantages of few techniques and exploiting compensations of the other. As the latest research focus is moved towards machine learning and deep learning models. Recommendation Systems are also being developed by employing these approaches, which are categorized as Model based approaches. Theoretically speaking they are those algorithms, which do the computing with predefined offline model, but presenting the result online. The design and development of models (such as machine learning, deep learning algorithms) allow the system to learn to recognize complex patterns based on the training data, and then make intelligent predictions for the collaborative filtering tasks for test data or real-world data, based on the learned models.However the input features could be content-based instead of ratings. 
\newline To find the relevant and needy information from the flood of data in a short chunk of time, increases the demand and popularity of recommendation systems.Now a wide variety of domains opting the development of these systems ranging from News to web-services. Most of the times these domains have textual data and may take advantage of the work done by different researchers for some other domain. So it is highly important to gather all such work in one place. Moreover, the trend of work in terms of computational technology has been also revived. Now deep learning models are replacing traditional models in almost every domain of computer sciences. This is why this study would be a good overview of the work done in recent past to the ones who are going to start the work in this area.
\newline With this work we compiled all the varieties of work done in last 2 decades to provide an overview of what has been achieved on textual data recommendations. How the algorithmic strategies has been evolved and what were most popular and available data sets. Moreover this study also gives a picture of How researchers are evaluating their work. The insights of this study will give an effective piece of information to the interested fellows.
\newline Rest of this paper is organized as follows,In section 3 methodology to conduct this study is presented in detail. Section 4 first discuss the popular evaluation metrics and their formulas, then a summary report is presented in table. X.The detailed description of all the data sets used in selected studies are described in section 5 followed by the consolidated table. Next section discussed the various methods and sources used in literature to extract features from textual data. In section 7 the algorithmic approaches for recommendations are discussed in detailed. Section 8 gives an overview of previously done surveys in this respect. The last section presents the overall summary table of this work and discussion about the insights are presented.