\subsection{Machine Learning Based Approaches}
Machine Learning is a useful tool used to solve learning problems in Computer Science field in recent years.
\cite{N5} recommends the journal and conferences to researchers to publish their papers by using the information of their abstract. They first use tf-idf to extract the feature set, after that Chi-square feature selection method is used forthe selection of most effective features.They used softmax regression for training and recommendation purposes.
\\CCCCCC There are totally 14,012 records containing title, abstract, author and link of papers. To ensure the records in the dataset are correct, 20 percent of abstracts from each journal and conference are checked manually. Two-thirds of all abstracts are used as training samples, and the rest are used for testing. In experiment, papers published in 2014 and 2013 are considered.
\\They have also published their work in as a web-service\footnote{http://www.keaml.cn/prs/}.
\\
\\
\cite{27}developed Sentiment Utility Logistic Model (SULM) to recommend items along with the useful aspect that may result on better user experience; based on the user reviews and ratings.The detailed features are extracted as aspects and sentiments extracting from user reviews.The model was trained using Stochastic Gradient Descent to learn to predict the unknown ratings and sentiments about various aspectsand also identifies the impacts of these aspects on the overall rating of the item. Moreover, we use these estimated impacts to recommend the most valuable aspects to the users to enhance their experiences with the recommended items. SULM thus goes one step further and significantly enhances the functionality of the current recommender systems by providing all these additional capabilities to the traditional rating prediction and recommendation tasks.